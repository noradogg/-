% ------------------------------------------------------------------------------
% thesisstyle.tex
% thesisstyle template 2014
% 2012/02/09 written by Kenzo OKUDA.
% 2014/02/10 updated by Kenzo OKUDA.
% 
% Copyright (C) 2012-2014 Kenzo OKUDA. 
% Advanced Communications and Networking Research Group, Osaka City University. 
% ------------------------------------------------------------------------------

\NeedsTeXFormat{pLaTeX2e}
\listfiles	% 処理したファイル一覧をlogに出力

% -- encoding
% use utf8 as input encodings
\usepackage[utf8]{inputenc}
% use T1(8 bits, PostScript Type 1) as font encodings
\usepackage[T1]{fontenc}
% support for the Text Companion fonts
\usepackage{textcomp}


% -- graphics
\usepackage[dvipdfmx]{graphicx,color}
% graphics search-path
\graphicspath{{figures/}{graphs/}{src/}{img/}}
% Patches for LaTeX
\usepackage{fixltx2e}	% LaTeX2e の図配置の問題を修正
% Emulation of obsolete package for "here" floats
\usepackage{here}	% 
% Improved interface for floating objects
% \usepackage{float}	% this is replaced by here.sty
% Execute command after the next page break
\usepackage{afterpage}
% getting BoundingBox from PDF Mediabox
% render extractbb/ebb/xbb unnecessary
% http://www.ma.ns.tcu.ac.jp/Pages/TeX/mediabb.sty.html
%\usepackage{mediabb}%	PDFから直接BBを取得する.ebbの実行が不要.


% -- sub figure
%% subfigure.sty
% obsolute.標準非搭載で使えない.
%\usepackage{subfigure}\def\subfloat{\subfigure}

%% subfig [2005/06/28 ver: 1.3 subfig package]
% 古い.標準搭載
%\usepackage[caption=false,font=footnotesize]{subfig}
%	\captionsetup[subfloat]{labelformat=simple,listofformat=simple,subrefformat=simple}
%	\renewcommand{\thesubfigure}{(\alph{subfigure})}
%	%図参照番号で副番号を括弧でくくる 11a -> 11(a)
%	\renewcommand{\thesubtable}{(\alph{subtable})}
%	%表参照番号で副番号を括弧でくくる 11a -> 11(a)
%	\captionsetup[subtable]{labelformat=simple,listofformat=simple,subrefformat=simple}
%	\def\subfigure{\subfloat}

%% subcaption [2013-02-03]
% 新しい.標準搭載.captionの一部.そろそろsubcaptionに乗り換えたい
%\usepackage[labelformat=simple]{subcaption}
%\renewcommand\thesubfigure{(\alph{subfigure})}

%% subfloat.sty	
% subfiure環境を実現する.
% subfiure環境でfigure環境を囲むとサブキャプションになる.
%\usepackage{subfloat}


% -- table
% Publication quality tables in LaTeX
\usepackage{booktabs}	% tabular環境で\topruleなどを使う.
% Extending the array and tabular environments
\usepackage{array}	% arrayとtabular環境を改善
% Create tabular cells spanning multiple rows
\usepackage{multirow}
% Create equal-widthed par­boxes
%\usepackage{eqparbox}	% 同じ幅に調整されるボックス


% -- math
% AMS mathematical facilities for LaTeX
\usepackage{amsmath}
% preventing page breaks from occurring within multiline equations
\interdisplaylinepenalty=2500
% TeX fonts from the American Mathematical Society
\usepackage{amssymb}
% Access bold symbols in maths mode
\usepackage{bm}


% -- font
% obsolete package
%  * \usepackage{times}
%  * \usepackage{pslatex}
% more old package
%  * \usepackage{txfonts}
%    * Times-like fonts in support of mathematics
% old package
%  * \usepackage{mathptmx}
%    *  Use Times as default text font, and provide maths support
% latest packages for using Times fonts
%  * \usepackage{newtxtext,newtxmath}
%    * Alternative uses of the TX fonts, with improved metrics
%
%\usepackage{txfonts}
%\usepackage{newtxtext,newtxmath}
%
% Arbitrary size font selection in LaTeX
\usepackage{type1cm}	% Computer Modernのサイズ変更を有効にする.


% -- mapfile for embedding fonts
% Base 14 fonts
%\AtBeginDvi{\special{pdf:mapfile psbase14.map}}
% Adobe base 14 fonts (Karl Berry names)
%\AtBeginDvi{\special{pdf:mapfile kbbase14.map}}
% Alternative base 14 fonts (urw fonts in Ghostscript)
%\AtBeginDvi{\special{pdf:mapfile dlbase14.map}}


% -- citation style
% Improved citation handling in LATEX
%\usepackage{cite}
% Flexible bibliography support
\usepackage[square, comma, numbers, sort&compress]{natbib}


% -- others
% Verbatim with URL-sensitive line breaks
\usepackage{url}
% Selectively include/excludes portions of text
\usepackage{comment}
% Control layout of itemize, enumerate, description
%\usepackage{enumitem}


%%
%% End of file `thesisstyle.tex'.

